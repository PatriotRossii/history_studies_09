\documentclass{article}
\usepackage[utf8]{inputenc}

\usepackage[T2A]{fontenc}
\usepackage[utf8]{inputenc}
\usepackage[russian]{babel}

\usepackage{multienum}
\usepackage{geometry}
\usepackage{hyperref}

\geometry{
    left=1cm,right=1cm,
    top=2cm,bottom=2cm
}

\usepackage{graphicx}
\graphicspath{ {./images/} }

\title{История}
\author{Лисид Лаконский}
\date{February 2023}

\newtheorem{definition}{Определение}

\begin{document}
\raggedright

\maketitle
\tableofcontents
\pagebreak

\section{Практическое занятие по истории №10, «Гражданская война в России и формирование большевистского строя»}

\subsection{Дискуссионные проблемы хронологии и периодизации гражданской войны в России}

 Проблема периодизации гражданской войны одна из дискуссионных. Впервые она запечатлена на страницах журнала «Вопросы истории» за 1954-1955 гг., где обсуждался вопрос о хронологических границах гражданской войны. Противоборствовали два предложения:

\begin{enumerate}
    \item \textbf{«объединить в один период Великую Октябрьскую социалистическую революцию и гражданскую войну, явившуюся продолжением революции, непосредственно следовавшую за ней»}, и назвать его «периодом победы социалистической революции и гражданской войны в СССР»(1917-1920гг.); с этим предложением выступили И.Б.Берхин и М.П.Ким;
    \item \textbf{установить раздельно} «мирный период социалистического строительства (октябрь 1917 г.- июль 1918 г.)» и «период иностранной военной интервенции и гражданской войны» (с лета 1918 г. до конца 1920 г.), не допуская стирания «грани» между ними (А.П.Кучкин).
\end{enumerate}

Большинство участников дискуссии \textbf{поддержали второе предложение}, считая, что военные вопросы (введение «военного коммунизма», строительство Красной Армии, организация отпора врагу) были главными лишь в период с весны – лета 1918 г. до конца 1920 г., т.е. с начала мятежа чехословацкого корпуса и белогвардейцев до разгрома армии Врангеля. Определяя свою позицию, они ссылались на учение Сталина и Ленина.

\hfill

Вместе с тем у этих теоретиков существовали различия во взглядах по этому вопросу. В частности, \textbf{И. В. Сталин} очерчивал рамки гражданской войны лишь периодом \textbf{с лета 1918 г. до конца 1920 г.}

\hfill

По мнению \textbf{В.И.Ленина}, гражданская война продолжалась \textbf{с 1917 г. по ноябрь 1922 г.}, но на протяжении этих лет вооруженная борьба имела не одинаковое значение в жизни общества. Из этого периода он исключал вооруженную борьбу в первые послеоктябрьские месяцы, считая, что боевые действия на этом этапе гражданской войны начались с победы над Керенским под Гатчиной, продолжились победами над буржуазией, юнкерами, частью контрреволюционного казачества в Москве, Иркутске, Оренбурге, Киеве и закончились победой над Калединым, Корниловым и Алексеевым в Ростове – на – Дону. Таким образом, Ленин назвал и основные события, и продолжительность гражданской войны на начальном этапе, и классовую характеристику противников Советской власти в первое время ее существования. Второй этап гражданской войны Ленин ограничивал 1918-1921гг. и характеризовал его как эпоху, когда «все внимание и все силы были устремлены или поглощены, главным образом, задачей отпора от нашествия, от опасности быть немедленно задушенными гигантскими силами мирового империализма». Рассматривая этот период в масштабе истории советского общества в целом, Ленин считал его «первым периодом Советской власти»и характеризовал как период «ожесточеннейшей гражданской войны и бешеного саботажа». После 1921г. задачи внешние и военные, по мнению Ленина, уже не стояли перед Советской властью как неотложные. Он писал: «Перед нами стоят сейчас, главным образом, экономические задачи».

Большинство современных историков \textbf{придерживаются ленинской периодизации} гражданской войны (например, В.Д.Поликарпов, И. И. Минц).

\hfill

Вместе с тем в литературе встречаются и иные мнения. Например, \textbf{некоторые авторы полагают, что гражданская война имела место с февраля 1917 г. по октябрь 1922 г.} Они объясняют свою позицию тем, что после свержения царизма в России наблюдался рост политического самосознания всех классов и социальных слоев общества, размежевание и консолидация классовых сил. Подтверждением этому стали апрельский и июльский 1917 г. кризисы Временного правительства, а корниловский мятеж представлял собой заговор, приведший к фактическому началу гражданской войны со стороны буржуазии. С точки зрения этих историков, второй период гражданской войны начинается с октября 1917 г. и продолжается по октябрь 1922 г.

\hfill

В 90-е годы в \textbf{институте Российской истории РАН} была предложена новая периодизация истории Гражданской войны в России. Она охватывает период с октября 1917 г. по 1922 г. и выделяет 6 этапов:

\begin{enumerate}
    \item \textbf{25 октября 1917 года – май 1918 года} - начало вооруженного гражданского противостояния;
    \item \textbf{май 1918 года – ноябрь 1918 года} - начало полномасштабной гражданской войны;
    \item \textbf{ноябрь 1918 года – весна 1919 года} - усиление противостояния «красных» и «белых»;
    \item \textbf{весна 1919 года – конец 1919 года} - разгром основных сил «белых»; эвакуация основных сил иностранных войск;
    \item \textbf{весна 1920 года – конец 1920 года} - война с Польшей, разгром армии Врангеля;
    \item \textbf{конец 1920 года – 1922 год} - победа «красных» в Средней Азии, Закавказье, на Дальнем Востоке, завершение гражданской войны. 
\end{enumerate}

\pagebreak
\subsection{Психология общества и отдельных социальных групп в годы Гражданской войны}

Особую роль в тенденциях в данной сфере, проявившихся в
годы Гражданской войны, сыграли изменения в сознании и по ведении массовых слоев общества, произошедшие в результате Первой мировой войны и революции 1917 г., которые стали предпосылками социокультурных сдвигов периода Гражданской войны. К ним можно отнести разочарование большинства населения в традиционной власти, изменение моральных установок в оценке пределов и допустимости насилия, коррозию религиозной веры; озлобление крестьянства против помещиков, хуторян, горожан; массовые настроения рабочих в пользу государственного регулирования производства и перераспределения ценностей.

\hfill

В период революции 1917 г. в условиях активизации пропаганды леворадикальных политических партий и групп был вызван к жизни психологический механизм эскалации ненависти низов по отношению к «внутренним врагам», помещикам и буржуазии, стимулировавшей акты социального возмездия. Ликвидация частной собственности, основ экономического неравенства, перераспределение богатств, насилие и принуждение по отношению к «классовым врагам» стали рассматриваться массами как необходимые компоненты и условие успешного движения к социализму, идеал которого укрепился в их сознании в ходе революции

\hfill

В России к весне 1918 г. кризисные процессы революционного развития, усугубленные длительным состоянием войны, привели к небывалой экономической разрухе, голоду и безработице в городах, дезорганизации транспорта, падению промышленного производства, натурализации народного хозяйства, и, что чрезвычайно важно, — неустойчивости власти, не признаваемой значительной частью населения, породившей своими действиями глубокий раскол и противостояние в обществе. В этих условиях сознание основной массы населения характеризовалось отличительными чертами, присущими «психопатологии Смуты»: крайней неустойчивостью, частыми переходами от страха к агрессии, стремлением нажиться за чужой счет, перераспределить собственность, склонностью к локализации негативных психологических комплексов на образе внутреннего врага, неуверенностью в завтрашнем дне (чему, кстати, не противоречила вера в мировую революцию,
призванная на психологическом уровне сыграть роль защитного механизма), одновременной готовностью к жертвенному подвигу,
смерти и желанием выжить во что бы то ни стало (что было во многом также наследием милитаризованного сознания, опыта, приобретенного на войне)

\hfill

Нам представляется обоснованной позиция В. В. Кабанова, который выделял в качестве фак торов изменения психологии крестьян опыт, полученный ими в борьбе за установление советской власти и против нее, всеобщую разруху, потери, болезни, экономическую и социальную поли тику советской власти, и, особенно, тотальную маргинализацию деревни. Отрицательный опыт быстро менял человека и имел далеко идущие последствия: утрата человеческого достоинства, страх, злоба и т. д. В то же время крестьянин, вернувшийся с войны, все же расширял свой кругозор, становился самостоя тельнее, уменьшалась власть над ним консервативных традиций, однако позитивные элементы в психологии тонули в море отрицательного опыта. Одновременно в эти годы рождалось новое влияние: духовное формирование новых социальных типов шло путем индокгринации, «напичкивания» лозунгами большевиков, соответственно примитивизируемыми. Как отмечал В. В. Каба нов, происходило «заглатывание» без пережевывания не очень здоровой духовной пищи, а то и ее суррогата. Неоднозначность изменений социальной психологии крестьянства под влиянием Гражданской войны отмечают и западные исследователи. Так, Ш. Фицпатрик, опираясь на данные О. Файджеса, пишет, что в деревнях ветераны Гражданской войны, несмотря на их молодость, пользовались авторитетом как люди, повидавшие мир, и часто выступали против крестьянского консерватизма, спорили со старшими по миру

\hfill

Ветераны Красной армии после Гражданской войны образовали костяк советской администрации. В 1926 — 1927 гг. более половины председателей сельсоветов и более двух третей председателей и членов волостных советов России были ветеранами Красной армии. Ветераны войны из рабочего класса выдвигались и в ряды управляющих промышленностью. Стиль организации и руководства ветеранов отражал их военный опыт и «штатские» коммунисты часто сетовали, что ими командуют вместо того, чтобы советоваться с ними и что они мало ценят демократические принципы советской власти. Н. К. Крупская писала о политвоспитательной работе в начале 1920-х гг.: «Когда гражданская война стала подходить к концу, в политпросвет влились громадные кадры военных работников, перенесшие в политпросветработу все методы работы на фронте. Самодеятельность населения, все формы углублен ной работы были сведены на нет»2. В целом, опираясь на данные исследователей, можно говорить о глубоком влиянии идеологии и психологии военного коммунизма и опыта гражданской войны на формирование политической культуры не только советской партийно-государственной элиты, но и всего советского общества

\hfill

В современной историографии убедительно обоснован тезис о широком применении насилия и террора обеими противоборствующими сторонами не только по отношению к вооруженному противнику, но и мирному населению. Мотивы мести, устрашения, наживы, аффективные состояния были психологическим контекстом поступков людей, вовлеченных в противостояние, по обе сто роны баррикад. Актуализация архаичных инстинктов, смещение нравственных ориентиров, моральное разложение было характер но для представителей как «белого», так и для «красного» лагерей

\pagebreak
\subsection{Политическое и военное противоборство в основных регионах России}

Читай по ссылке: \href{https://foxford.ru/wiki/istoriya/sovetskaya-rossiya-v-period-grazhdanskoy-voyny}{Советская Россия в период Гражданской войны, Фоксфорд}

\pagebreak
\subsection{Доктринальные основы большевистского строя}

Политическая доктрина большевиков представляла собой сумму идей, заимствованных у народников и К. Маркса. У народников была взята идея прямой демократии (самоуправления народа), из марксизма — идея диктатуры пролетариата и будущего безгосударствснного коммунистического строя. Буржуазное общество было для них неприемлемо не только с точки зрения марксисткой теории: все они по многу лет жили в эмиграции в Европе, видели реальность. И прекрасно знали, что критические оценки западноевропейского строя высказывали и многие властители дум, далекие от марксизма. Так, еще Г. Ф. Гегель писал: «В гражданском обществе каждый для себя — цель, все остальное для него ничто»[5]. Или: «Гражданское общество является ареной борьбы частных индивидуальных интересов, войны всех против всех»[6]. Поэтому большевики стремились к созданию не правового, а социального государства.

\hfill

После прихода к власти большевиков уровень прав и свобод в стране многократно вырос. Естественно, это были не нрава буржуазного общества, а воплощение в жизнь народнической системы самоуправления. Большевики стремились создать общество социальной справедливости в соответствии с особенностями национальной ментальности. В течение ноября 1917 г. — января 1918 г. они юридически оформили систему самоуправления народа:

\begin{enumerate}
    \item основу государственного механизма составили декреты «О праве отзыва делегатов» (4 декабря 1917 г.), «О Высшем Совете Народного Хозяйства» (14 декабря 1917 г.), «Об отделении церкви от государства и школы от церкви» (2 февраля 1918 г.), «О Рабоче-крестьянской Красной Армии» (28 января 1918 г.), «О приобретении прав российского гражданства» (1 апреля 1918 г.) и др.;
    \item экономические основы нового общественного строя регулировали «Положение о рабочем контроле» (27 ноября 1917 г.), декреты «О национализации банков» (27 декабря 1917 г.), «О социализации земли» (9 февраля 1918 г.), «О национализации внешней торговли» (22 апреля 1918 г.), «О национализации крупнейших предприятий горной, металлургической, металлообрабатывающей, текстильной и других ведущих отраслей промышленности» (28 июля 1918 г.);
    \item основы социальной системы закладывали декрет ВЦИК «Об уничтожении сословий и гражданских чинов» (23 ноября 1917 г.) и др.;
    \item основы национальной политики были заложены в «Декларации прав народов России» (15 ноября 1917 г.), обращении СНК «Ко всем трудящимся мусульманам России и Востока» (3 декабря 1917 г)., постановлении СНК «О Финляндской республике» (31 декабря 1917 г.) и др.
\end{enumerate}

Идея правой демократии стала основой «Декларация прав трудящегося и эксплуатируемого народа» (18 (31) января 1918 г.) и Конституции 1918 г.

\hfill

Однако к осени 1918 г. от созданной большевиками системы самоуправления народа ничего не осталось. На территории, которую они контролировали, власть оказалась в руках местных комитетов РКП(б) и работавших под их руководством ревкомов. Сами большевики считали, что созданный ими политический режим явился результатом «гражданской войны и разорения», и определяли его как «военный коммунизм» (т.е. вынужденный). В действительности же его внутренняя природа и причины были более сложными, чем их воспринимали его создатели.

\hfill

Исследований последних лег позволяет предположить, что основная причина ликвидации демократических завоеваний 1917 г. в течение первой половины 1918 г. была связана не с доктринальными установками большевиков (курсом на установление диктатуры пролетариата, ликвидацией частной собственности), а с отсутствием гражданского общества на местном, региональном уровне.

\hfill

Начавшийся весной 1917 г. процесс развала экономики, государства и страны к весне 1918 г. привел к коллапсу. В этих условиях большевики весной — летом 1918 г. ликвидировала систему Советов как основу местного сепаратизма и развернули борьбу с политическими оппонентами. Осенью 1917 г. — весной 1918 г. ими оказались кадеты, потом эсеры и меньшевики, со второй половины 1918 г. — белое движение, церковь, интеллигенция.

\hfill

К концу Гражданской войны демократия исчезла уже в самой большевистской партии. В 1920 г. в РКП(б) возникли две фракции: «Рабочая оппозиция» и «Демократический централизм». Суть их требований состояла в возвращении к дореволюционному политическому идеалу — народовластию. В. И. Ленин не выступал против самой идеи, но подчеркивал, что это результат длительного исторического развития. На теоретическом уровне в 1921 г. ситуацию попытался объяснить II. И. Бухарин:

«Формы демократии идут в убывающем порядке в зависимости от двух обстоятельств: во-первых, от социально-классового состава данной организации..., во-вторых, этот демократизм еще определяется в зависимости от функций тех аппаратов, которые имеются в виду, чем больше функций воспитательных и функций лабораторно-мыслительных (подготовка се и т.д.), тем больше демократизм необходим, и наоборот, чем больше административных функций, тем меньше демократизма»[8].

\hfill

Этого объяснения вполне хватало с политической точки зрения, но совершенно не хватало с теоретической: требовалась разработка теории новой реальности. Но сложнейшие теоретически проблемы были сведены к внутрипартийным перебранкам.

\hfill

К 1923 г. недовольство внутрипартийными отношениями разрослось до реальной внутрипартийной борьбы. К этому времени за работу партийного аппарата и кадровую политику отвечал И. В. Сталин. В 1923 г. в ответ на выступления Л. Д. Троцкого и «Заявление 46» он язвительно заметил:

«Я далек от того, чтобы отрицать значение перевыборов под углом зрения демократизма в деле улучшения нашей внутрипартийной жизни. Но видеть в этом основную гарантию — значит не понимать ни внутрипартийной жизни, ни ее недочетов.

В рядах оппозиции имеются такие, как Белобородов, “демократизм” которого до сих пор остается в памяти у ростовских рабочих; Розенгольц, от “демократизма” которого не поздоровилось нашим водникам и железнодорожникам; Пятаков, от “демократизма” которого не кричал, а выл весь Донбасс; Альский, “демократизм” которого всем известен; Бык, от “демократизма” которого воет Хорезм. Думает ли Сапронов, что если нынешних “партийных педантов” сменят поименованные выше “уважаемые товарищи”, демократия внутри партии восторжествует? Да будет мне позволено несколько усомниться в этом».

\pagebreak
\subsection{Основные этапы формирования большевистского строя}

Большевизм существует как течение политической мысли и как политическая партия с 1903 г. Характер большевизма был обусловлен той полемикой, которая зародилась в самом начале его формирования. Благодаря гениальной ясности мысли, уверенной настойчивости и полемическому темпераменту Ленин сыграл в ней выдающуюся роль. Прежде чем открылся съезд, были одержаны победы в трех идеологических битвах. В отличие от народников Российская социал-демократическая рабочая партия считала движущей силой грядущей революции пролетариат, а не крестьянство. В отличие от «легальных марксистов» она призывала к революционным действиям и борьбе за социализм. В отличие от так называемых «экономистов» она выдвигала от имени пролетариата не только экономические, но и политические требования. Борьба с народниками была главным достижением Плеханова.

\subsubsection{Политический режим}

Понятие «политический режим» выражает характер взаимосвязи государственной власти и индивида.

\hfill

Совокупность средств и методов, используемых государством при отправлении власти, отражает степень политической свободы в обществе и правовое положение личности. В зависимости от степени социальной свободы индивида и характера взаимоотношений государства и гражданского общества различают три типа режимов: тоталитарный, авторитарный и демократический. Между демократией и тоталитаризмом, как крайними полюсами данной классификации, располагается множество промежуточных форм власти. Например, полудемократические режимы характеризуются тем, что фактическая власть лиц, занимающих лидирующие позиции, заметно ограничена, а свобода и демократичность выборов настолько сомнительны, что их результаты заметно расходятся с волей большинства. Кроме того, гражданские и политические свободы урезаны настолько, что организованное выражение политических целей и интересов просто невозможно.

\hfill

Политическая система - это «набор» политических институтов и отношений, в рамках которого осуществляется власть и обеспечивается ее стабильность, а политический режим - способ функционирования политической системы общества, определяющий характер политической жизни в стране, отражающий уровень политической свободы и отношение органов власти к правовым основам их деятельности.

\hfill

Политический режим определяется уровнем развития и интенсивностью общественно-политических процессов, структурой правящей элиты, механизмом ее формирования, состоянием свобод и прав человека в обществе, состоянием отношений с бюрократией, господствующим в обществе типом легитимности, развитостью общественно-политических традиций, доминирующим в обществе политическим сознанием и поведением.

\hfill

Смена политической системы, как правило, приводит к смене политического режима, что и произошло в России в 1917 году.

\subsubsection{Ленинско-большевистский политический режим}

Февральская революция 1917 года установила республиканский политический режим без его оформления конституционным образом. Законный правопреемник власти отсутствовал, и наступил этап выявления наиболее адекватного условиям России носителя власти. Утвердившееся благодаря поддержке Советов рабочих и солдатских депутатов Временное правительство, где доминировали либералы, а позднее вошли меньшевики и эсеры, не решалось взять на себя ответственность ни за наделение крестьян землей, ни за окончание войны, ни за созыв Учредительного собрания. Ни Государственное совещание, ни Демократическое совещание, ни Предпарламент, которые созывал А.Ф. Керенский, не могли заменить Учредительное собрание. Несмотря на то, что Временное правительство в качестве истока имело комитет Государственной думы, оно не обладало настоящей легитимностью. Кроме того, длительная зависимость от советов рабочих депутатов изначально заложила в массах недоверие к нему как псевдолегитимному непостоянному органу власти.

\hfill

Февральский политический режим носил переходный характер и должен был закончиться установлением либо правой военной, либо левой коммунистической диктатурой - это были две реальные альтернативы осени 1917 года.

\hfill

Государственный кризис принял цивилизационный характер. Февральское низвержение Николая II означало конец всей системы самодержавия, которая была на протяжении всей тысячелетней истории Руси-России стержневой цивилизационной основой нации. Существовала достаточно реальная опасность установления всеобщей анархии и бунта всех против всех. В этих условиях было совершенно недостаточно наличия каких-то радикальных партий и решительных лидеров, чтобы спасти страну от хаоса и беспредела. На помощь пришла сама история Российской цивилизации, предложившая политикам возможность использования традиционных для русского народа соборных форм соучастия во власти: соборы, круги, общины, собрания, вече, сходки, советы и др. Зародившиеся в июле 1905 г. стихийным образом Советы были ничем иным, как своеобразным проявлением соборной традиции русского народа искать сообща выход из тяжелой ситуации. Большевики первоначально отнеслись весьма настороженно к органам управления в форме Советов депутатов трудящихся, считая их своими возможными конкурентами, но В.И. Ленин в августе 1905 года первым оценил колоссальные возможности Советов для организации новой системы власти под началом большевизма. Кроме того, Ленин увидел в Советах прообраз общественного устройства, которое будет не государством, а средством объединения трудящихся в грядущем бесклассовом обществе. Жизнь показала, что второй аспект оказался преждевременным и утопическим, как и вся концепция мировой революции, которой придерживался тогда лидер большевизма.

\hfill

Зародившаяся советская система власти была изначально подлинно народной, исторически обоснованной и логичной, продолжением соборной традиции. Поэтому не случайно Советы стихийно возникли во всех городах страны с различными схемами выборов.

\hfill

Октябрьская революция не положила начало непосредственной мировой революции, но она, несомненно, простимулировала глобальные реформистские преобразования на Западе, в результате которых трудящиеся добились значительных социальных завоеваний, а сам капитализм впоследствии принял весьма цивилизованный вид общества «социального партнерства».

\hfill

Большевики всеми силами обеспечивали большинство в Советах рабочим и членам партийной элиты как самой пролетарской, в результате чего советская власть начала приобретать черты однопартийной диктатуры. Главным орудием строительства новой государственности был Совет Народных Комиссаров во главе с В.И. Лениным, который с самого начала освободился от контроля Советов и начал формирование специфического большевистского политического режима власти.

\hfill

Контуры советской государственности определялись первой Конституцией РСФСР, принятой в июле 1918 года, которая одновременно стала самой первой конституцией в России в целом. Основной закон отразил влияние недавней революции и начинавшейся гражданской войны. Бывшие эксплуататоры лишались гражданских прав, исключались из политической жизни нетрудовые элементы и предусматривались неравные права для избирателей города и села. Выборы были многостепенными, что обеспечивало нужный состав всех Советов. Хотя высшими органами власти считались ВЦИК, съезд Советов и СНК, на самом деле значительно больше полномочий имел СНК - правительство РСФСР. Однако фактически действительно высшими политическими органами власти в стране были ЦК РКП (б) и Политбюро.

\hfill

Государственное строительство развивалось в годы гражданской войны под влиянием задач вооруженной борьбы и социально-классового противоборства на всех уровнях общественной жизни. В этот период отмечается рост централистских и милитаристских тенденций, быстрое увеличение чрезвычайных органов во многих жизненно важных сферах.

\hfill

В полосе боевых действий создавались ревкомы, заменявшие Советы и проводившие чрезвычайные меры. В целом, несмотря на свою громоздкость, дублирование отдельных элементов, государственный аппарат оказался достаточно работоспособным и обеспечил условия для победы большевизма в гражданской войне. Ведущую роль в этом сыграло наличие квалифицированной политической элиты, ленинской «старой партийной гвардии», получившей в прошлом определенную образовательную и профессиональную подготовку, опыт политической деятельности и боевую закалку. Следует отметить особое значение совпадения характера, личных качеств людей, стоявших во главе революционного движения с его характером.

\hfill

Несмотря на условия гражданской войны, а может быть благодаря им, внутри большевистского политического режима соблюдались определенные нормы относительной демократии и товарищеские взаимоотношения. Эта характеристика ленинского режима прослеживается в годы осуществления новой экономической политики, либерализации хозяйственных отношений в обществе и становления рыночного механизма. Однако данная тенденция начинает активно свертываться с середины 20-х гг. и заменяться противоположной, авторитарно-бюрократической.

\subsubsection{Сталинско-большевистский политический режим}

После ухода В.И. Ленина с политической арены развернулась острая внутрипартийная борьба, принявшая внутриэлитный и личностный характер. Началось утверждение режима личной власти И.В. Сталина, что привело к формированию в рамках советско-коммунистической системы новой вариации большевистского политического режима.

\hfill

Можно определить его в отличие от ленинско-большевистского как сталинско-большевистский политический режим. Несмотря на преемственность и общие черты режимов совершенно очевидны и отличия достаточно принципиального свойства.

\hfill

В период осуществления новой экономической политики, разрешившей развитие частнособственнических тенденций в экономике и узаконившей экономический плюрализм, режим диктатуры большевизма не носил законченного тоталитарного характера. В рамки нэпа умещалась электрификация народного хозяйства, трудовая кооперация, начало культурной революции, планирование экономики и сближение трудящихся классов. Но плюрализм в экономике и диктатура в политике не могли развиваться параллельно, так как были несовместимы в перспективе, и поэтому объективно требовалось либо введение многопартийности в соответствии с многоукладностью экономики и отказ большевизма от диктатуры, либо ее укрепление и ликвидация нэповского экономического плюрализма. Логика развития страны как осажденной крепости предопределила второй вариант развития политической системы.

\hfill

Традиционные для крестьянской ментальности царистско-вождистские ориентации были в полной мере и абсолютно сознательно использованы И.В. Сталиным, сформировавшим с помощью пропаганды культ своей личности как непогрешимого вождя. Было официально объявлено, что в СССР построен в основном социализм, что не соответствовало действительности, так как социалистические идеалы народовластия были далеки от осуществления. Наряду с этим существовали отдельные элементы социалистического характера, например общественно-государственная собственность на средства производства. В трудовых коллективах и массовом сознании всего общества воспроизводились в модернизированной коммунистической форме традиционные принципы общности, солидарности, приоритета нематериальных стимулов. Масштаб распространения в народе социалистического сознания закономерно связывался с тем, что, во-первых, среди населения испокон века зиждились духовно-нравственные ценности справедливости, добра, патриотизма, коллективизма, духовности. Во-вторых, кризис церкви и развитие атеизма лишали православную религию возможности контролировать эгалитаристские тенденции нации.

\hfill

Не случайно большевизм трактуется некоторыми исследователями как хилиазм христианства или ересь православия. Н.А. Бердяев справедливо считал советско-коммунистический строй естественным следствием всей истории России.

\hfill

Сталинско-большевистский режим объективно продолжил дело форсированной индустриализации, начатой еще при Витте в дореволюционный период и при всех эксцессах обеспечил форсированное вхождение страны в индустриальное общество.

\hfill

При всех недостатках и пороках сталинский режим объективно выполнил главную задачу - спасения во время второй мировой войны всего человечества от фашизма, что было признано союзниками СССР - западными демократиями в лице их лидеров - Черчилля и Рузвельта.

\hfill

В годы Великой Отечественной войны советская государственность дополнилась новыми компонентами, прежде всего внедрением в коммунистическую идеологию национально-патриотических лозунгов. Сталин разочаровался в идее мировой революции и распустил Коминтерн, отказался от политики воинствующего атеизма и признал исторические духовно-культурные ценности России великим достоянием СССР. В итоге возник тот сплав приверженности трудящихся советскому социалистическому строю с многовековыми патриотическими традициями русских и других народов нашей родины, который и обеспечил победу. Однако сталинский режим несет и свою долю ответственности за неоправданно высокие потери. Хотя на фронтах погибло 8,5 миллионов военнослужащих, что сопоставимо с потерями интервентов, но, кроме того, фашистами было истреблено 20 миллионов мирных жителей и военнопленных, а в итоге было уничтожено целое послеоктябрьское поколение советских людей, что имело далеко идущие последствия.

\hfill

В этот период чрезмерная централизация режима, наконец, получила известное оправдание. В то же время, по мнению ряда историков, в годы войны произошло ослабление диктаторского режима и повышение степени самостоятельности и ответственности исполнителей на местах. Следует подчеркнуть, что И.В. Сталин несёт персональную ответственность за многие поражения и просчеты начального этапа войны, но он неотделим как Верховный Главнокомандующий и от побед заключительного этапа. Блестящую объективную характеристику Сталину дал Черчилль в своей известной речи в британском парламенте. Великую и страшную фигуру Сталина невозможно рассматривать вне реальной истории советского государства со всеми его победами и поражениями.

\subsection{Эволюция политической системы в годы Гражданской войны}

\hfill

\subsection{Политическая система РСФСР в конце 1920 — начале 1921 г.}

\subsubsection{Экономический и социальный кризис конца 1920 г. – начала 1921 г.}

Политика «военного коммунизма» привела экономику страны к полному развалу. С ее помощью не удалось преодолеть разруху, порожденную 4 годами участия России в Первой мировой войне и усугубленную 3 годами Гражданской войны. Население уменьшилось на 10,9 млн. человек. Во время военных действий особенно пострадали Донбасс, Бакинский нефтяной район, Урал и Сибирь, были разрушены многие шахты и рудники. Из-за нехватки топлива и сырья останавливались заводы. Рабочие были вынуждены покидать города и уезжали в деревню. Петроград потерял 60\% рабочих, когда закрылись Путиловский, Обуховский и другие предприятия, Москва — 50\%. Прекратилось движение на 30 железных дорогах. Безудержно нарастала инфляция. Сельскохозяйственной продукции производилось только 60\% довоенного объема. Посевные площади сократились на 25\%, так как крестьяне не были заинтересованы в расширении хозяйства. В 1921 г. из-за неурожая массовый голод охватил город и деревню.

\hfill

Провал политики «военного коммунизма» большевистское правительство осознало не сразу. В 1920 г. Совнарком продолжил меры по усилению безрыночных, распределительно-коммунистических начал. Национализация промышленности была распространена на мелкие предприятия. В декабре 1920 г. VIII Всероссийский съезд Советов утвердил План восстановлении народного хозяйства и его электрификации (план ГОЭЛРО). В феврале 1921 г. Совнарком создал Государственную комиссию (Госплан) для разработки текущих и перспективных планов хозяйственного развития страны. Расширился ассортимент продуктов сельского хозяйства, подлежащих продразверстке. Готовился декрет об отмене денежного обращения. Однако эти мероприятия вступали в противоречие с требованиями рабочих и крестьян. Параллельно с экономическим в стране нарастал социальный кризис.

\hfill

Рабочих раздражали безработица и нехватка продуктов питания. Они были недовольны ущемлением прав профсоюзов, введением принудительного труда и его уравнительной оплаты. Поэтому в городах в конце 1920 г. — начале 1921 г. начались забастовки, в которых рабочие выступали за демократизацию политической системы страны, созыв Учредительного собрания, отмену спецраспределителей и пайков. Крестьяне, возмущенные действиями продотрядов, перестали не только сдавать хлеб по продразверстке, но и поднялись на вооруженную борьбу. Восстания охватили Тамбовщину, Украину, Дон, Кубань, Поволжье и Сибирь.

\hfill

Крестьяне требовали изменения аграрной политики, ликвидации диктата РКП(б), созыва Учредительного собрания на основе всеобщего равного избирательного права. На подавление этих выступлений были брошены части Красной Армии и ВЧК.

\subsubsection{Восстание в Кронштадте}

В марте 1921 г. моряки и красноармейцы военно-морской крепости Кронштадт потребовали освобождения из заключения всех представителей социалистических партий, проведения перевыборов Советов и изгнания из них коммунистов, предоставления свободы слова, собраний и союзов всем партиям, обеспечения свободы торговли, разрешения крестьянам свободно пользоваться землей и распоряжаться продуктами своего хозяйства, т. е. ликвидации продразверстки. Кронштадтцев поддержали рабочие. В ответ правительство ввело осадное положение в Петрограде, объявило восставших мятежниками и отказалось вести с ними переговоры. Полки Красной Армии, усиленные отрядами ВЧК и делегатами X съезда РКП(б), специально прибывшими из Москвы, штурмом взяли Кронштадт. 2,5 тыс. матросов было арестовано, 6—8 тыс. эмигрировало в Финляндию.

\hfill

Разруха и голод, забастовки рабочих, восстания крестьян и матросов — все свидетельствовало о том, что в стране назрел глубокий экономический и социальный кризис. Кроме того, к весне 1921 г. была исчерпана надежда на скорую мировую революцию и материально-техническую помощь европейского пролетариата. Поэтому В. И. Ленин пересмотрел внутриполитический курс и признал, что только удовлетворение требований крестьянства может спасти класть большевиков.

\end{document}