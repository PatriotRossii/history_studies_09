\documentclass{article}
\usepackage[utf8]{inputenc}

\usepackage[T2A]{fontenc}
\usepackage[utf8]{inputenc}
\usepackage[russian]{babel}

\usepackage{multienum}
\usepackage{geometry}
\usepackage{hyperref}

\geometry{
    left=1cm,right=1cm,
    top=2cm,bottom=2cm
}

\usepackage{graphicx}
\graphicspath{ {./images/} }

\title{История}
\author{Лисид Лаконский}
\date{February 2023}

\newtheorem{definition}{Определение}

\begin{document}
\raggedright

\maketitle
\tableofcontents
\pagebreak

\section{Практическое занятие по истории №10, «Гражданская война в России и формирование большевистского строя»}

\subsection{Дискуссионные проблемы хронологии и периодизации гражданской войны в России}

\subsection{Психология общества и отдельных социальных групп в годы Гражданской войны}

\subsection{Политическое и военное противоборство в основных регионах России}

\subsection{Доктринальные основы большевистского строя}

\subsection{Основные этапы формирования большевистского строя}

\subsection{Эволюция политической системы в годы Гражданской войны}

\subsection{Политическая система РСФСР в конце 1920 — начале 1921 г.}

\end{document}